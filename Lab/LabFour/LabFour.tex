\documentclass[10pt]{article}

%defines page size and margins
\usepackage{geometry}
\geometry{
    letterpaper,
    left=1in,
    right=1in,
    top=1in,
    bottom=1in,
}

%Sets spacing for entire document
\usepackage{setspace}
\singlespacing

%Package for reducing space in between list items
\usepackage{enumitem}

%Math symbols
\usepackage{gensymb}
\usepackage{siunitx}

%Image path
\usepackage{graphicx}
\graphicspath{ {/} }

%Used for adjusting images
\usepackage[export]{adjustbox}

%For floating images
\usepackage{float}

\begin{document}
\title{Laboratory Four --- OP Amps}
\date{November 3, 2017}
\author{Rishabh Shah\\ 4655 4192\\ \\ Partner: Matthew Remillard}
\maketitle
\newpage

\section*{Pre-Lab}

\section*{Lab Data}
\subsection*{Voltage Follower}
\begin{table}[H]
	\centering
	\begin{tabular}{lll}
		\hline
		$V_{in}$ & $V_{CC}$ & $V_{out}$\\
		\hline
		$4.997V$ & $8.00V$ & $3.369V$\\
		\hline
	\end{tabular}
	\caption{Voltage follower data}
\end{table}

\subsection*{Inverting Amplifier}
$R_X = 58475 \Omega$
\begin{figure}[H]
	\centering
	\includegraphics[width=\textwidth]{InvertingAmp.png}
	\caption{$V_{in}$ and $V_{out}$ vs Time for the inverting amplifier}
\end{figure}

\subsection*{Non-Inverting Amplifier}
$R_X = 105820 \Omega$
\begin{figure}[H]
	\centering
	\includegraphics[width=\textwidth]{NonInvertingAmp.png}
	\caption{$V_{in}$ and $V_{out}$ vs Time for the non-inverting amplifier}
\end{figure}

\subsection*{Clipping}
\begin{table}[H]
	\centering
	\begin{tabular}{llllll}
		\hline
		$+V_{CC}(V)$ & $-V_{CC}(V)$ & $V_{OUT,MAX}(V)$ & $V_{OUT,MIN}(V)$ & $\Delta V+(V)$ & $\Delta V-(V)$\\
		\hline
		15 & -15 & 13.1 & -13.3 & 1.9 & -1.7\\
		20 & -20 & 17.7 & -17.9 & 2.3 & -2.1\\
		\hline
	\end{tabular}
	\caption{Data when $R_L=2k\Omega$}
\end{table}
\begin{figure}[H]
	\centering
	\includegraphics[width=\textwidth]{Clipping20V1.png}
	\caption{$V_{in}$ and $V_{out}$ vs Time for when $R_L=2k\Omega$}
\end{figure}
\begin{table}[H]
	\centering
	\begin{tabular}{llllll}
		\hline
		$+V_{CC}(V)$ & $-V_{CC}(V)$ & $V_{OUT,MAX}(V)$ & $V_{OUT,MIN}(V)$ & $\Delta V+(V)$ & $\Delta V-(V)$\\
		\hline
		15 & -15 & 13.3 & -13.7 & 1.7 & -1.3\\
		20 & -20 & 18.1 & -18.3 & 1.9 & -1.7\\
		\hline
	\end{tabular}
	\caption{Data when $R_L=10k\Omega$}
\end{table}
\begin{figure}[H]
	\centering
	\includegraphics[width=\textwidth]{Clipping20V2.png}
	\caption{$V_{in}$ and $V_{out}$ vs Time for when $R_L=10k\Omega$}
\end{figure}

\subsection*{Phase Shift and Time Delay}
\begin{table}[H]
	\centering
	\begin{tabular}{lll}
		\hline
		Frequency $(kHz)$ & Shift $(\mu s)$ & Shift $(\degree)$\\
		\hline
		2 & -13.3 & 5.38\\
		5 & -11.6 & -20.5\\
		10 & -9.6 & -36.3\\
		20 & -7.9 & -57.2\\
		\hline
	\end{tabular}
	\caption{Data for phase shift and time delay}
\end{table}
\begin{figure}[H]
	\centering
	\includegraphics[width=\textwidth]{PhaseShift2.png}
	\caption{$V_{in}$ and $V_{out}$ vs Time for when frequency is $2kHz$}
\end{figure}
\begin{figure}[H]
	\centering
	\includegraphics[width=\textwidth]{PhaseShift5.png}
	\caption{$V_{in}$ and $V_{out}$ vs Time for when frequency is $5kHz$}
\end{figure}
\begin{figure}[H]
	\centering
	\includegraphics[width=\textwidth]{PhaseShift10.png}
	\caption{$V_{in}$ and $V_{out}$ vs Time for when frequency is $10kHz$}
\end{figure}
\begin{figure}[H]
	\centering
	\includegraphics[width=\textwidth]{PhaseShift20.png}
	\caption{$V_{in}$ and $V_{out}$ vs Time for when frequency is $20kHz$}
\end{figure}
\end{document}